%% LyX 2.3.5.2 created this file.  For more info, see http://www.lyx.org/.
%% Do not edit unless you really know what you are doing.
\documentclass[english]{article}
\usepackage[T1]{fontenc}
\usepackage[latin9]{inputenc}
\usepackage{amsmath}
\usepackage{amsthm}

\makeatletter
%%%%%%%%%%%%%%%%%%%%%%%%%%%%%% Textclass specific LaTeX commands.
\theoremstyle{plain}
\newtheorem{thm}{\protect\theoremname}
\theoremstyle{plain}
\newtheorem{prop}[thm]{\protect\propositionname}

\makeatother

\usepackage{babel}
\providecommand{\propositionname}{Proposition}
\providecommand{\theoremname}{Theorem}

\begin{document}
\begin{prop}
If $k\left(\theta\right)>0$, then it must be that $\alpha\theta>R$.
Furthermore, $\phi>0$, and $c_{1}^{y}=c_{1}^{0}+\frac{\beta\phi}{\lambda_{1}\left(1-\alpha\right)}$.
\end{prop}

\begin{proof}
The proof proceeds by contradiction. Assume that for some $\theta$,
$k\left(\theta\right)>0$ at the optimum of the planner's problem,
but $\alpha\theta<R$. Then, from the first-order condition for $k$
in the planner's problem, 
\[
R-\alpha\theta=\frac{\mu}{\lambda_{1}\theta c_{0}}-\frac{\phi}{\lambda_{1}\left(c_{0}+k\right)}>0
\]
which implies that 
\[
\frac{\mu}{\lambda_{1}\theta c_{0}}>\frac{\phi}{\lambda_{1}\left(c_{0}+k\right)}>0
\]
Thus, $\mu$--the multiplier on the incentive constraint--is positive.
Then, rearranging the first-order conditions for $c_{0}$ and $k$
gives
\begin{align*}
R+\frac{\phi}{\lambda_{1}\left(c_{0}+k\right)}= & \frac{c_{1}^{y}}{\beta c_{0}}-\frac{\mu k}{\lambda_{1}\theta c_{0}^{2}}\\
R+\frac{\phi}{\lambda_{1}\left(c_{0}+k\right)}= & \alpha\theta+\frac{\mu}{\lambda_{1}\theta c_{0}}
\end{align*}
which implies
\[
R+\frac{\phi}{\lambda_{1}\left(c_{0}+k\right)}=\frac{c_{1}^{y}}{\beta c_{0}}-\frac{k}{c_{0}}\left(\underbrace{R+\frac{\phi}{\lambda_{1}\left(c_{0}+k\right)}-\alpha\theta}_{>0}\right)
\]
The term in parentheses on the right-hand side is equal to $\frac{\mu}{\lambda_{1}\theta c_{0}}$,
which by the above, is positive. This implies that 
\[
\frac{c_{1}^{y}}{\beta c_{0}}>R+\frac{\phi}{\lambda_{1}\left(c_{0}+k\right)}
\]
This, however, implies that $\mu<0$, contradicting the assumption.
Thus, $k\left(\theta\right)>0\implies\alpha\theta>R$. This also shows
that if the first-order condition for $k\left(\theta\right)$ in the
planner's problem holds, $\mu<0$. 

The first-order condition for $c_{1}\left(\theta,0\right)$, meanwhile,
can be rearranged to give 
\[
c_{1}^{y}=c_{1}^{0}+\frac{\beta\phi}{\lambda_{1}\left(1-\alpha\right)}
\]
By definition, $\phi\ge0$, with equality if the constraint in REF
does not hold. Equation REF ABOVE shows that $k>0\implies\phi>0$;
otherwise, the agents would have no incentive to bear the risk of
investing. 
\end{proof}
\begin{prop}
If $k>0$, $\tau_{k}\ge0$, with equality if $\theta=\overline{\theta}$,
and $\tau_{b}>0$. If $k=0$, $\tau_{b}=0$. 
\end{prop}

\begin{proof}
Recall that the intertemporal wedges are given by 
\begin{align*}
\tau_{k}\left(\theta\right)= & 1-\frac{c_{1}^{y}}{\alpha\beta\theta c_{0}}\\
\tau_{b}\left(\theta\right)= & 1-\frac{\frac{1}{c_{0}}}{\beta R\left(\frac{\alpha}{c_{1}^{y}}+\frac{1-\alpha}{c_{1}^{0}}\right)}
\end{align*}
Begin with the case where $k>0$. Combining the first-order conditions
in the planner's problem for $c_{0}$ and $k$ gives 
\[
\frac{c_{1}^{y}}{\beta c_{0}}-\frac{\mu k}{\lambda_{1}\theta c_{0}^{2}}=\alpha\theta+\frac{\mu}{\lambda_{1}\theta c_{0}}
\]
which implies
\begin{align*}
\alpha\theta-\frac{c_{1}^{y}}{\beta c_{0}} & =\frac{\mu}{\lambda_{1}\theta c_{0}}\left(-\frac{k}{c_{0}}-1\right)\\
\tau_{k}=1-\frac{c_{1}^{y}}{\alpha\beta\theta c_{0}} & =\frac{\mu}{\lambda_{1}\alpha\theta^{2}c_{0}}\left(-\frac{k}{c_{0}}-1\right)
\end{align*}
Recall that if $k\left(\theta\right)>0$, $\mu\left(\theta\right)<0$.
Because all other variables on the right-hand side of the final equality
are nonnegative, the above shows that $\tau_{k}>0$. Turning to the
wedge on risk-free savings, note that if $k\left(\theta\right)>0$,
then period-1 consumption $c_{1}\left(\theta\right)$ becomes a random
variable, so by Jensen's inequality, 
\begin{align*}
\frac{\alpha}{c_{1}^{y}}+\frac{1-\alpha}{c_{1}^{0}} & >\frac{1}{\alpha c_{1}^{y}+\left(1-\alpha\right)c_{1}^{0}}\implies\\
\alpha c_{1}^{y}+\left(1-\alpha\right)c_{1}^{0} & >\frac{1}{\frac{\alpha}{c_{1}^{y}}+\frac{1-\alpha}{c_{1}^{0}}}\implies\\
\frac{\alpha c_{1}^{y}+\left(1-\alpha\right)c_{1}^{0}}{\beta Rc_{0}} & >\frac{\frac{1}{c_{0}}}{\beta R\left(\frac{\alpha}{c_{1}^{y}}+\frac{1-\alpha}{c_{1}^{0}}\right)}
\end{align*}
So in order to establish a positive wedge $\tau_{b}$, it is sufficient
to show that $\beta Rc_{0}>\alpha c_{1}^{y}+\left(1-\alpha\right)c_{1}^{0}$.
From the first-order conditions in the planner's problem, 
\begin{align*}
\alpha c_{1}^{y}+\left(1-\alpha\right)c_{1}^{0} & =c_{1}^{y}-\frac{\beta\phi}{\lambda_{1}}
\end{align*}
and
\[
\beta Rc_{0}=c_{1}^{y}-\frac{\beta\mu k}{\lambda_{1}\theta c_{0}}-\frac{\beta\phi c_{0}}{\lambda_{1}\left(c_{0}+k\right)}
\]
Combining and multiplying through by $\left(-\beta/\lambda_{1}\right)$,
it suffices to show that 
\[
\phi\left(1-\frac{c_{0}}{c_{0}+k}\right)\ge\frac{\mu k}{\theta c_{0}}
\]
Note, however, that this inequality holds trivially: by definition,
the right hand side is positive, while the left is negative.

In the case where $k\left(\theta\right)=0$, $c_{1}\left(\theta,y\right)=c_{1}\left(\theta,0\right)\equiv c_{1}\left(\theta\right)$;
the planner does not find it optimal for these types to invest, and
so she has no need to incentivize them to do so. By Proposition REF1,
then, $\phi\left(\theta\right)=0$. Then, the first-order condition
for $c_{0}\left(\theta\right)$ in the planner's problem gives
\[
1=\frac{c_{1}}{\beta Rc_{0}}
\]
and thus $\tau_{b}\left(\theta\right)=0$ for types who do not invest. 
\end{proof}

\end{document}
