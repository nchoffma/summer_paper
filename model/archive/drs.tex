\documentclass[11pt]{article}
\usepackage[margin = 1in]{geometry}
\usepackage{amsmath}
\usepackage{amssymb}
\usepackage{amsthm} % for proof environment
\usepackage{enumitem}
\usepackage{graphicx}
\usepackage{indentfirst}
\usepackage{caption}
\usepackage{lscape}
\usepackage{multirow}
\usepackage{array}
\usepackage{setspace}
\setlist{nolistsep}
\usepackage[round]{natbib}
\usepackage{accents}
\usepackage{caption}
\usepackage{subcaption}

\newcommand{\ubar}[1]{\underaccent{\bar}{#1}}
\newcommand{\p}{\prime}
\newcommand{\ev}{\mathbb{E}}
\newcommand{\lagr}{\mathcal{L}}
\newcommand{\inv}[1]{#1^{-1}}
\newcommand{\R}{{\rm I\!R}}
\newcommand{\U}{\mathcal{U}}
\renewcommand{\H}{\mathcal{H}}
\newcommand{\pderiv}[2]{\frac{\partial#1}{\partial #2}}

\begin{document}
    \begin{flushleft}
        Optimal Taxation with Heterogeneous Rates of Return \\
        Returns to Scale \\
        \today
    \end{flushleft}

\section{Constant and Decreasing Returns to Scale} 

I consider again a two-type version of the model, where \( \theta\in\{\ubar{\theta}, \bar{\theta}\} \). I compare two specifications for the production function: \( y = \theta k \) and \( y = \theta k^\alpha \), where \( \alpha < 1 \). In both specifications, agents can borrow and lend in the first period at rate \( R > 1 \). I assume that \( \theta_L < R < \theta_H \).

The first case, wherein production exhibits constant returns to scale, is identical to the previous two-type case. With this specification, both the individual agents in the non-distorted equilibrium and the planner with information and resource constraints found it optimal to set \( k(\theta_L) = 0 \), and have the \( \theta_L \)-type agents lend to the \( \theta_H \)-types. The second case, in which production exhibits \textit{decreasing} returns to scale, admits a different solution. In this case, in the non-distorted competitive equilibrium, both types will be on their Euler equations for capital and bonds; denoting \( f(k) = \theta k^\alpha \), an agent of type \( \theta \) will choose \( k \) and \( b \) such that \( f^\p(k) = R \). Denoting the optimal tax schedule over capital income as \( T(f(k),Rb) \), and \( T_1 \) and \( T_2 \) its first and second partial derivatives respectively, this system will be such that the after-tax returns are equated,
\begin{equation}
    (1 - T_1)f^\p(k) = (1 - T_2)Rb
\end{equation}

\section{Computational Results}



\end{document}