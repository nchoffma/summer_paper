%% LyX 2.3.5.2 created this file.  For more info, see http://www.lyx.org/.
%% Do not edit unless you really know what you are doing.
\documentclass[english]{article}
\usepackage[T1]{fontenc}
\usepackage[latin9]{luainputenc}
\usepackage{amsmath}
\usepackage{amssymb}
\usepackage{babel}
\begin{document}
\title{Fixed Probability Model: Proofs (Scratch)}

\maketitle
FOCs that the algorithm solves for:
\begin{align*}
c_{0}: & R=\frac{c_{1}^{y}}{\beta c_{0}}-\frac{\mu k}{\lambda_{1}\theta c_{0}^{2}}-\frac{\phi}{\lambda_{1}\left(c_{0}+k\right)}\\
c_{1}^{0}: & 1=\frac{c_{1}^{y}}{c_{1}^{0}}-\frac{\beta\phi}{\lambda_{1}\left(1-\alpha\right)c_{1}^{0}}\\
k: & R=\alpha\theta+\frac{\mu}{\lambda_{1}\theta c_{0}}-\frac{\phi}{\lambda_{1}\left(c_{0}+k\right)}\\
\eta: & \mathcal{U}=\log c_{0}+\beta\left(\alpha u\left(c_{1}^{y}\right)+\left(1-\alpha\right)u\left(c_{1}^{0}\right)\right)\\
\phi: & \mathcal{U}\left(\underline{\theta}\right)=\log\left(c_{0}+k\right)+\beta\log\left(c_{1}^{0}\right)
\end{align*}

First thing: computationally, it is the case that $k=0$ if $\alpha\theta<R$.
Why?

A couple of things to note: 
\[
c_{1}^{y}=c_{1}^{0}+\frac{\beta\phi}{\lambda_{1}\left(1-\alpha\right)}
\]
 So, $\phi$ (plus some parameters) describes how incentives are provided.
If $k>0$, $\phi>0$, so $\phi$ describes how risky investment is
incentivized. if $k=0$, then $c_{1}^{y}=c_{1}^{0}$; these types
do not need to be incentivized. 

WTS: $k\left(\theta\right)>0\iff\alpha\theta>R$

$\implies$: assume $k>0$, but $\alpha\theta<R$. From FOC for $k$:
if $\alpha\theta<R$, then 
\begin{align*}
R-\alpha\theta=\frac{\mu}{\lambda_{1}\theta c_{0}}- & \frac{\phi}{\lambda_{1}\left(c_{0}+k\right)}>0\\
\frac{\mu}{\lambda_{1}\theta c_{0}}> & \frac{\phi}{\lambda_{1}\left(c_{0}+k\right)}>0\text{ by definition}
\end{align*}
 which implies that if these two hold, $\mu>0$. 

Rearranging the FOCs for $c_{0}$ and $k$:
\begin{align*}
R+\frac{\phi}{\lambda_{1}\left(c_{0}+k\right)}= & \frac{c_{1}^{y}}{\beta c_{0}}-\frac{\mu k}{\lambda_{1}\theta c_{0}^{2}}\\
R+\frac{\phi}{\lambda_{1}\left(c_{0}+k\right)}= & \alpha\theta+\frac{\mu}{\lambda_{1}\theta c_{0}}
\end{align*}
which gives
\[
R+\frac{\phi}{\lambda_{1}\left(c_{0}+k\right)}=\frac{c_{1}^{y}}{\beta c_{0}}-\frac{k}{c_{0}}\left(\underbrace{R+\frac{\phi}{\lambda_{1}\left(c_{0}+k\right)}-\alpha\theta}_{\ge0?}\right)
\]
 The bracketed term is equal to $\frac{\mu}{\lambda_{1}\theta c_{0}}$,
which as shown above, is positive (if both assumptions hold). This
implies that 
\[
\frac{c_{1}^{y}}{\beta c_{0}}>R+\frac{\phi}{\lambda_{1}\left(c_{0}+k\right)}
\]
 which implies that $\mu<0$, a contradiction. Thus, $k>0\implies\alpha\theta>R$. 

$\impliedby$: Assume $\alpha\theta>R$, but $k=0$. Then, the FOC
for $k$ does not hold, which makes this difficult. This also means
that $\phi=0$. In that case, we get the following allocations:
\begin{align*}
R= & \frac{c_{1}^{y}}{\beta c_{0}}\\
c_{1}^{y}= & c_{1}^{0}\\
\mathcal{U}= & \log c_{0}+\beta\log c_{1}
\end{align*}

I think this is the issue: it IS the case that $k>0\implies\alpha\theta>R$.
But, it does \textbf{not} go the other way. Computationally, the cutoff
is \textbf{at least} $R/\alpha$, but in practice, it can be higher
depending on $\lambda_{1}$.

Next: what can we say about the remaining allocations? Intuitively
(prove/disprove):
\begin{enumerate}
\item The larger the optimal $k$, the larger the distance must be between
$c_{1}^{y}$ and $c_{1}^{0}$. This would imply that $\phi$ is increasing
in $k$. 
\item The logical pattern is that (at the optimum): $k,c_{1}^{y}$ increasing
in $\theta$, $c_{1}^{0}$ decreasing in $\theta$ (must be the case
if $c_{1}^{y}$ is increasing, in order to balance out the final constraint). 
\item What is less clear is the pattern of $c_{0}$: in some of the Matlab
examples, $c_{0}$ is decreasing in $\theta$. 
\item Furthermore, the wedges are
\begin{align*}
\tau_{k}\left(\theta\right)= & 1-\frac{c_{1}^{y}}{\alpha\beta\theta c_{0}}\\
\tau_{b}\left(\theta\right)= & 1-\frac{\frac{1}{c_{0}}}{\beta R\left(\frac{\alpha}{c_{1}^{y}}+\frac{1-\alpha}{c_{1}^{0}}\right)}
\end{align*}
 What are the patterns in these?
\item Computationally: it looks like $\mu$ decreases in $\theta$ until
it hits the investment cutoff
\end{enumerate}
Working through:
\[
\tau_{k}=\frac{\mu}{\lambda_{1}\alpha\theta^{2}c_{0}}\left(\frac{-k}{c_{0}}-1\right)
\]
 This would suggest that $\tau$ is positive: if $k>0$, $\mu<0$
(by proof above), and so $\tau_{k}$ should be $>0$. Whether this
is increasing, or not, depends on $\mu^{\prime},c_{0}^{\prime},k^{\prime}$.
This also implies the ``zero top marginal rate'' result, as $\mu\left(\underline{\theta}\right)=0$. 

The complication here is that each of these equations has multiple
unknowns in $\theta$, so the derivatives are somewhat intertwined.
The place to start, it seems, is this:
\[
c_{1}^{y}\left(\theta\right)=c_{1}^{0}\left(\theta\right)+\frac{\beta\phi\left(\theta\right)}{\lambda_{1}\left(1-\alpha\right)}
\]
 So intuitively, $\phi$ represents the spread between $c_{1}^{y}$
and $c_{1}^{0}$. It also represents the shadow value of relaxing
the $\mathcal{U}\left(\underline{\theta}\right)$ constraint, which
intuitively I think would increase in $\theta$: for higher values
of $\theta$, there is more to gain from relaxing this constraint,
allowing them to invest more. 

Other issue: from computational results, it's not obvious that any
of these are monotonic.

What about the wedge for those who do not invest? Their FOCs are
\begin{align*}
R= & \frac{c_{1}}{\beta c_{0}}\\
\mathcal{U}= & \log c_{0}+\beta\log\left(c_{1}\right)\\
= & \log c_{0}+\beta\log\left(R\beta c_{0}\right)\\
= & \log\left(R\beta c_{0}^{\beta+1}\right)
\end{align*}

Wedge on risk-free asset:
\begin{align*}
\tau_{b}\left(\theta\right)= & 1-\frac{1}{c_{0}}\left(\frac{c_{1}}{\beta R}\right)\\
= & 1-\frac{c_{1}}{\beta Rc_{0}}\\
= & 0
\end{align*}
This checks out computationally. 

If $k>0$, all of the following hold: 
\begin{align*}
c_{0}: & R=\frac{c_{1}^{y}}{\beta c_{0}}-\frac{\mu k}{\lambda_{1}\theta c_{0}^{2}}-\frac{\phi}{\lambda_{1}\left(c_{0}+k\right)}\\
c_{1}^{0}: & 1=\frac{c_{1}^{y}}{c_{1}^{0}}-\frac{\beta\phi}{\lambda_{1}\left(1-\alpha\right)c_{1}^{0}}\\
k: & R=\alpha\theta+\frac{\mu}{\lambda_{1}\theta c_{0}}-\frac{\phi}{\lambda_{1}\left(c_{0}+k\right)}\\
\eta: & \mathcal{U}=\log c_{0}+\beta\left(\alpha u\left(c_{1}^{y}\right)+\left(1-\alpha\right)u\left(c_{1}^{0}\right)\right)\\
\phi: & \mathcal{U}\left(\underline{\theta}\right)=\log\left(c_{0}+k\right)+\beta\log\left(c_{1}^{0}\right)
\end{align*}
 Idea: higher $k$ means higher variance in second-period consumption,
measured by the spread between $c_{1}^{y}$ and $c_{1}^{0}$. This
in turn depends on $\phi$. 
\[
\phi=\frac{\lambda_{1}}{\beta}\left(1-\alpha\right)\left(c_{1}^{y}-c_{1}^{0}\right)
\]

Risk-free wedge:
\[
\tau_{b}\left(\theta\right)=1-\frac{\frac{1}{c_{0}}}{\beta R\left(\frac{\alpha}{c_{1}^{y}}+\frac{1-\alpha}{c_{1}^{0}}\right)}
\]
 If $k>0$, $c_{1}$ is a random variable:
\[
\mathbb{E}\left[\frac{1}{c_{1}}\right]=\frac{\alpha}{c_{1}^{y}}+\frac{1-\alpha}{c_{1}^{0}}\ge\frac{1}{\mathbb{E}\left[c_{1}\right]}=\frac{1}{\alpha c_{1}^{y}+\left(1-\alpha\right)c_{1}^{0}}
\]
 by Jensen's inequality. 

What else would we like to prove about this?
\begin{enumerate}
\item Still would like to show $\phi$ increasing in $k$. 
\item Ideally, we would like to show that types $\theta<\overline{\theta}$
invest $k>0$, for the right value of $\alpha$.
\begin{enumerate}
\item Possible steps:
\begin{enumerate}
\item What is the limit on how much capital one can invest? It seems like
such a limit is introduced by the final constraint
\item Given this limit, it seems like we can improve on the ``only one
type invests'' allocations by bumping up 
\end{enumerate}
\end{enumerate}
\end{enumerate}

\end{document}
