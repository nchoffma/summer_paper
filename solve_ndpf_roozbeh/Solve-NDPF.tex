%% LyX 2.2.0 created this file.  For more info, see http://www.lyx.org/.
%% Do not edit unless you really know what you are doing.
\documentclass[english]{article}
\usepackage[T1]{fontenc}
\usepackage[latin9]{inputenc}
\usepackage{geometry}
\geometry{verbose,tmargin=1in,bmargin=1in,lmargin=1in,rmargin=1in}
\usepackage{amsmath}
\usepackage{amsthm}

\makeatletter
%%%%%%%%%%%%%%%%%%%%%%%%%%%%%% Textclass specific LaTeX commands.
  \theoremstyle{plain}
  \newtheorem*{assumption*}{\protect\assumptionname}
  \theoremstyle{remark}
  \newtheorem*{note*}{\protect\notename}
\theoremstyle{plain}
\newtheorem{thm}{\protect\theoremname}

%%%%%%%%%%%%%%%%%%%%%%%%%%%%%% User specified LaTeX commands.
\usepackage{geometry}
\usepackage{multirow}
\geometry{verbose,letterpaper,tmargin=1in,bmargin=1in,lmargin=1in,rmargin=1in}
\setlength{\parskip}{0pt}
\usepackage{setspace}
\onehalfspacing
\makeatletter
\usepackage{amsfonts}
\usepackage{latexsym}
\usepackage{color}
\usepackage[bookmarks=false,colorlinks,citecolor=navy,linkcolor=maroon,urlcolor=maroon,backref]{hyperref}

\newtheorem{theorem}{Theorem}
\newtheorem{acknowledgement}[theorem]{Acknowledgement}
\newtheorem{algorithm}[theorem]{Algorithm}
\newtheorem{axiom}[theorem]{Axiom}
\newtheorem{case}[theorem]{Case}
\newtheorem{claim}[theorem]{Claim}
\newtheorem{conclusion}[theorem]{Conclusion}
\newtheorem{condition}[theorem]{Condition}
\newtheorem{conjecture}{Conjecture}
\newtheorem{corollary}[theorem]{Corollary}
\newtheorem{criterion}[theorem]{Criterion}
\newtheorem{definition}{Definition}
\newtheorem{example}[theorem]{Example}
\newtheorem{exercise}[theorem]{Exercise}
\newtheorem{lemma}{Lemma}
\newtheorem{notation}[theorem]{Notation}
\newtheorem{problem}[theorem]{Problem}
\newtheorem{proposition}{Proposition}
\newtheorem{remark}[theorem]{Remark}
\newtheorem{solution}[theorem]{Solution}
\newtheorem{summary}[theorem]{Summary}
\newtheorem{assumption}{Assumption}
%\newenvironment{proof}[1][Proof]{\noindent\textbf{#1.} }{\ \rule{0.5em}{0.5em}\\}
\def\argmax{\mathop{\rm arg\,max}}
\definecolor{Green}{rgb}{.3,.7,0}
\definecolor{orange}{rgb}{.7,.3,0}
\definecolor{maroon}{cmyk}{0,1,1,.3}
\definecolor{navy}{cmyk}{1,1,0,.1}
\usepackage[labelsep=colon,font={footnotesize},labelfont={sc},justification=justified]{caption} 

\makeatother

\makeatother

\usepackage{babel}
  \providecommand{\assumptionname}{Assumption}
  \providecommand{\notename}{Note}
\providecommand{\theoremname}{Theorem}

\begin{document}

\title{Solving Mirrleesian Optimal Taxation Problems with Infinitely Many
Types Using Finite Element Method}

\author{Roozbeh Hosseini}

\date{\today}
\maketitle

\section{Static Model}

Consider the following environment. Individual preferences are\footnote{There is nothing special about additive separability. All procedures
can be extended to more general preferences. } 
\[
U\left(c,l\right)=u\left(c\right)-v\left(l\right)
\]

Technology
\[
y=\theta l.
\]
Here $l$ is hours worked (or effort), $\theta$ is labor ability
(or productivity), and $y$ is effective labor services.
\begin{assumption*}
Only $y$ is observable by taxing authorities. Therefore, taxes cannot
be functions of $\theta$ or $l$.
\end{assumption*}
Labor ability/productivity $\theta$ has CDF $F(\theta)$ (PDF $f(\theta)$)
on $\left[\underline{\theta},\overline{\theta}\right]$ where $\bar{\theta}$
can be infinity. In what follows, I refer to $\theta$ as individual's
'type'.
\begin{note*}
In some of the derivations it is useful/convenient to use $y/\theta$
to denote hours worked.

Let $T\left(y\right)$ be a tax function. Then individual of type
$\theta$ faces the following decision problem
\begin{align*}
U\left(\theta\right) & =\max_{c,l}u\left(c\right)-v\left(\frac{y}{\theta}\right)\\
c & =y-T\left(y\right)
\end{align*}
Note that
\[
\theta u'\left(c\right)=\left(1-T'\left(y\right)\right)v'\left(\frac{y}{\theta}\right)
\]
Also, by envelope theorem
\[
\dot{U}\left(\theta\right)=\frac{y}{\theta^{2}}v'\left(\frac{y}{\theta}\right)
\]
Which we can rewrite as
\begin{equation}
U'\left(\theta\right)=\frac{l\left(\theta\right)}{\theta}v'\left(l\left(\theta\right)\right)\label{eq:Static-IC}
\end{equation}
Constraint \eqref{eq:Static-IC} is call \emph{implementability constraint
}(or \emph{incentive} \emph{compatibility constraint}).

Suppose there is government that has expenditure $G$. The government
budget constraint is 
\[
G=\int_{\theta}T\left(y\left(\theta\right)\right)f\left(\theta\right)d\theta.
\]
Finally allocation in the economy is feasible if 
\[
\int_{\theta}c\left(\theta\right)f\left(\theta\right)d\theta+G=\int_{\theta}y\left(\theta\right)f\left(\theta\right)d\theta.
\]
\end{note*}
\begin{thm}
Any feasible allocation $\left(c\left(\theta\right),l\left(\theta\right)\right)$
can be implemented via some income tax function $T\left(y\right)$
iff it satisfy implementability constraint \eqref{eq:Static-IC}.
\end{thm}
\begin{proof}
The necessity is obvious (outlined above). The sufficiency is by construction
of a tax function.
\end{proof}
This theorem transforms the problem of finding optimal policy function,
$T\left(y\right)$, (which is a very complicated problem) to a constrained
maximization problem over allocations (which can be solved using standard
methods).

\subsection{Planning Problem}

Consider the problem of a government who seeks to find policies that
maximize weighted average of welfare in the economy. Suppose government
assigns weight $g\left(\theta\right)$ to individual of type $\theta$.

Instead of writing this maximization problem over the set of policy
functions, we write the following maximization problem over the set
of \emph{implementable allocations}.

\[
\max\int_{\underline{\theta}}^{\overline{\theta}}U\left(\theta\right)g\left(\theta\right)f(\theta)d\theta
\]
s.t.
\[
G+\int_{\underline{\theta}}^{\overline{\theta}}\left(c(\theta)-\theta l(\theta)\right)f(\theta)d\theta=0\quad;\lambda
\]
\[
U(\theta)=u(c(\theta))-v\left(l(\theta)\right)\quad;f\left(\theta\right)\eta(\theta)
\]
\[
U'=\frac{l(\theta)}{\theta}v'\left(l(\theta)\right)\quad;\mu(\theta)f\left(\theta\right)
\]

First order conditions:
\begin{equation}
-\lambda+u'(c(\theta))\eta(\theta)=0\label{eq:FOC-c}
\end{equation}
\begin{equation}
\theta\lambda-\eta(\theta)v'\left(l(\theta)\right)+\frac{\mu(\theta)}{\theta}\left(v'\left(l(\theta)\right)+l(\theta)v''\left(l(\theta)\right)\right)=0\label{eq:FOC-l}
\end{equation}

Hamiltonian:
\begin{equation}
g(\theta)-\eta(\theta)+\mu'(\theta)+\frac{f'(\theta)}{f(\theta)}\mu(\theta)=0\label{eq:FOC-U}
\end{equation}

Boundary conditions:
\[
\mu(\overline{\theta})=\mu(\underline{\theta})=0
\]
Use \eqref{eq:FOC-c} to eliminate $\eta(\theta)$
\[
\theta-\frac{v'\left(l(\theta)\right)}{u'(c(\theta))}+\frac{\mu(\theta)}{\lambda\theta}\left(v'\left(l(\theta)\right)+l(\theta)v''\left(l(\theta)\right)\right)=0
\]
\[
g(\theta)+\frac{f'(\theta)}{f(\theta)}\mu(\theta)-\frac{\lambda}{u'(c(\theta))}+\dot{\mu}(\theta)=0
\]

We need to solve the following system of equations:
\begin{equation}
G+\int_{\underline{\theta}}^{\overline{\theta}}\left(c(\theta)-\theta l(\theta)\right)f(\theta)d\theta=0\label{eq:PK}
\end{equation}
\begin{equation}
U(\theta)=u(c(\theta))-v\left(l(\theta)\right)\label{eq:U}
\end{equation}
\begin{equation}
U'(\theta)=\frac{l(\theta)}{\theta}v'\left(l(\theta)\right)\label{eq:IC}
\end{equation}
\begin{equation}
\theta-\frac{v'\left(l(\theta)\right)}{u'(c(\theta))}+\frac{\mu(\theta)}{\lambda\theta}\left(v'\left(l(\theta)\right)+l(\theta)v''\left(l(\theta)\right)\right)=0\label{eq:FOC-lc}
\end{equation}
\begin{equation}
g(\theta)+\frac{f'(\theta)}{f(\theta)}\mu(\theta)-\frac{\lambda}{u'(c(\theta))}+\mu'(\theta)=0\label{eq:FOC-mu}
\end{equation}
\begin{equation}
\mu(\overline{\theta})=\mu(\underline{\theta})=0\label{eq:boundary}
\end{equation}
To solve for the following five: $c(\theta),l(\theta),U(\theta),\mu(\theta),\lambda$. 

Note that this is in fact an ODE in $U\left(\theta\right)$ and $\mu\left(\theta\right)$
with boundary conditions $\mu(\overline{\theta})=\mu(\underline{\theta})=0$.
So we can use method of weighted residual to solve it.

\subsection{Example:}

Consider the following example
\[
U(c,l)=\frac{c^{1-\sigma}}{1-\sigma}-\psi\frac{l^{\gamma}}{\gamma}
\]
So equations \eqref{eq:IC}, \eqref{eq:FOC-lc} and \eqref{eq:FOC-mu}
become
\[
U'\left(\theta\right)=\frac{\psi l\left(\theta\right)^{\gamma}}{\theta}
\]
\[
\theta-\psi l\left(\theta\right)^{\gamma-1}c\left(\theta\right)^{\sigma}+\frac{\mu(\theta)}{\lambda\theta}\psi\gamma l\left(\theta\right)^{\gamma-1}=0
\]
\[
g(\theta)+\frac{f'(\theta)}{f(\theta)}\mu(\theta)-\lambda c(\theta)^{\sigma}+\mu'(\theta)=0
\]
Take $\lambda$ as given. We want to solve the following system of
equations
\begin{eqnarray*}
U' & = & \psi\frac{l^{\gamma}}{\theta}\\
\mu' & = & \lambda c^{\sigma}-g-\frac{f'}{f}\mu
\end{eqnarray*}
where $l$ and $c$ are solutions to the following equation
\begin{eqnarray*}
\frac{c^{1-\sigma}}{1-\sigma}-\psi\frac{l^{\gamma}}{\gamma}-U & = & 0\\
\theta-\psi l^{\gamma-1}c^{\sigma}+\frac{\mu}{\lambda\theta}\psi\gamma l^{\gamma-1} & = & 0
\end{eqnarray*}

We approximate $\mu$ and $U$ with
\begin{eqnarray*}
U\left(\theta\right) & = & \sum_{n=1}^{N}\alpha_{n}\psi_{n}\left(\theta\right)\\
\mu\left(\theta\right) & = & \sum_{n=1}^{N}\beta_{n}\psi_{n}\left(\theta\right)
\end{eqnarray*}
where $\phi_{n}\left(\theta\right)$ is the tent function on $\left[\theta_{n-1},\theta_{n+1}\right]$. 

Define 
\begin{align*}
R_{\alpha}\left(\theta\right) & =U'\left(\theta\right)-\psi\frac{l\left(\theta;U,\mu,\lambda\right)^{\gamma}}{\theta},\\
R_{\beta}\left(\theta\right) & =\mu'\left(\theta\right)-\left(\lambda c\left(\theta;U,\mu,\lambda\right)^{\sigma}-g-\frac{f'}{f}\mu\right).
\end{align*}
We form the following system equations
\begin{align*}
\int_{\underline{\theta}}^{\bar{\theta}}\psi_{n}\left(\theta\right)R_{\alpha}\left(\theta\right)d\theta & =0\quad n=1,\dots,N\\
\int_{\underline{\theta}}^{\bar{\theta}}\psi_{n}\left(\theta\right)R_{\beta}\left(\theta\right)d\theta & =0\quad n=1,\dots,N\\
G+\int_{\underline{\theta}}^{\overline{\theta}}\left(c\left(\theta;U,\mu,\lambda\right)-\theta l\left(\theta;U,\mu,\lambda\right)\right)f(\theta)d\theta & =0
\end{align*}

This is a system of $2N+1$ equations to solve for $\alpha_{n}$,
$\beta_{n}$ and $\lambda$. The good news is each equations (except
the last one) in only relevant only on one interval $\left[\theta_{n},\theta_{n+1}\right]$.

Here is how the algorithm works:
\begin{enumerate}
\item Start with a guess of $\lambda$, $\alpha_{n}$ and $\beta_{n}$.
\item For $\theta\in\left[\theta_{n},\theta_{n+1}\right]$, find $U\left(\theta\right)$,
$\mu\left(\theta\right)$.
\item Solve for $c\left(\theta;U,\mu,\lambda\right)$ and $l\left(\theta;U,\mu,\lambda\right)$
such that
\begin{eqnarray*}
\frac{c^{1-\sigma}}{1-\sigma}-\psi\frac{l^{\gamma}}{\gamma}-U & = & 0,\\
\theta-\psi l^{\gamma-1}c^{\sigma}+\frac{\mu}{\lambda\theta}\psi\gamma l^{\gamma-1} & = & 0.
\end{eqnarray*}
These are just promise keeping and FOC w.r.t to $l$.
\item Evaluate $R_{\alpha}\left(\theta\right)$, $R_{\beta}\left(\theta\right)$
and feasibility.
\item Evaluate the derivative of the above equations w.r.t $\alpha_{n}$,
$\beta_{n}$ and $\lambda$.
\item Do the newton update.
\end{enumerate}
The following is useful in doing steps 2, 3, 4 and 5.

Let $\epsilon=2(\theta-\theta_{n})/(\theta_{n+1}-\theta_{n})-1$ and
$\Delta_{n}=\theta_{n+1}-\theta_{n}$. Then on the interval $\left[\theta_{n},\theta_{n+1}\right]$
\begin{eqnarray*}
U\left(\theta\right) & = & 0.5\alpha_{n}\left(1-\epsilon\right)+0.5\alpha_{n+1}\left(1+\epsilon\right)\\
\mu\left(\theta\right) & = & 0.5\beta_{n}\left(1-\epsilon\right)+0.5\beta_{n+1}\left(1+\epsilon\right)
\end{eqnarray*}
and 
\begin{eqnarray*}
U'\left(\theta\right) & = & \frac{-\alpha_{n}+\alpha_{n+1}}{\Delta_{n}}\\
\mu'\left(\theta\right) & = & \frac{-\beta_{n}+\beta_{n+1}}{\Delta_{n}}
\end{eqnarray*}
Therefore, we need to solve the system of $2N$ nonlinear equations
for $\alpha_{n}$and $\beta_{n}$
\begin{eqnarray*}
\frac{-\alpha_{n}+\alpha_{n+1}}{\Delta_{n}} & -\phi & \frac{l\left(\theta;\alpha_{n},\alpha_{n+1},\beta_{n},\beta_{n+1}\right)^{\gamma}}{\theta}=0\\
\frac{-\beta_{n}+\beta_{n+1}}{\Delta_{n}} & - & \left(\lambda c\left(\theta;\alpha_{n},\alpha_{n+1},\beta_{n},\beta_{n+1}\right)^{\sigma}-g-\left(0.5\beta_{n}\left(1-\epsilon\right)+0.5\beta_{n+1}\left(1+\epsilon\right)\right)\frac{f'}{f}\right)=0
\end{eqnarray*}
With conditions that $\alpha_{1}=\alpha_{N}=0$.

Derivative of the $n$th equation with respect to 
\begin{itemize}
\item $\alpha_{n}$
\[
-1/\Delta_{n}-\frac{\gamma\psi l^{\gamma-1}}{\theta}\frac{\partial l}{\partial\alpha_{n}}
\]
\[
-\sigma\lambda c^{\sigma-1}\frac{\partial c}{\partial\alpha_{n}}
\]
\item $\alpha_{n+1}$
\[
1/\Delta_{n}-\frac{\gamma\psi l^{\gamma-1}}{\theta}\frac{\partial l}{\partial\alpha_{n+1}}
\]
\[
-\sigma f\lambda c^{\sigma-1}\frac{\partial c}{\partial\alpha_{n+1}}
\]
\item $\beta_{n}$
\[
-\frac{\gamma\psi l^{\gamma-1}}{\theta}\frac{\partial l}{\partial\beta_{n}}
\]
\[
-1/\Delta_{n}-\left(\sigma\lambda c^{\sigma-1}\frac{\partial c}{\partial\beta_{n}}-0.5\left(1-\epsilon\right)\frac{f'}{f}\right)
\]
\item $\beta_{n+1}$
\[
-\frac{\gamma\psi l^{\gamma-1}}{\theta}\frac{\partial l}{\partial\beta_{n+1}}
\]
\[
1/\Delta_{n}-\left(\sigma\lambda c^{\sigma-1}\frac{\partial c}{\partial\beta_{n+1}}-0.5\left(1+\epsilon\right)\frac{f'}{f}\right)
\]
\end{itemize}
Now, we can use the promise keeping and IC to find $\frac{\partial c}{\partial}$
and $\frac{\partial l}{\partial}$
\[
\left[\begin{array}{ll}
c^{-\sigma} & -\psi l^{\gamma-1}\\
-\sigma\psi l^{\gamma-1}c^{\sigma-1} & \psi\left(\gamma-1\right)\left(\frac{\mu\gamma}{\lambda\theta}-c^{\sigma}\right)l^{\gamma-2}
\end{array}\right]\left[\begin{array}{c}
\frac{\partial c}{\partial U}\\
\frac{\partial l}{\partial U}
\end{array}\right]=\left[\begin{array}{c}
1\\
0
\end{array}\right]
\]
\[
\left[\begin{array}{ll}
c^{-\sigma} & -\psi l^{\gamma-1}\\
-\sigma\psi l^{\gamma-1}c^{\sigma-1} & \psi\left(\gamma-1\right)\left(\frac{\mu\gamma}{\lambda\theta}-c^{\sigma}\right)l^{\gamma-2}
\end{array}\right]\left[\begin{array}{c}
\frac{\partial c}{\partial\mu}\\
\frac{\partial l}{\partial\mu}
\end{array}\right]=\left[\begin{array}{l}
0\\
-\frac{\psi\gamma}{\lambda\theta}l^{\gamma-1}
\end{array}\right]
\]

Finally
\[
\frac{\partial U}{\partial\alpha_{n}}=0.5\left(1-\epsilon\right)
\]
\[
\frac{\partial U}{\partial\alpha_{n+1}}=0.5\left(1+\epsilon\right)
\]
\[
\frac{\partial\mu}{\partial\beta_{n}}=0.5\left(1-\epsilon\right)
\]
\[
\frac{\partial\mu}{\partial\beta_{n+1}}=0.5\left(1+\epsilon\right)
\]

\end{document}
