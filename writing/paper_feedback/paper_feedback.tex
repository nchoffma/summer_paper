\documentclass[11pt]{article}
\usepackage[margin = 1in]{geometry}

\title{Feedback on Optimal Taxation}
\author{}
\date{}

\begin{document}

\maketitle

\section{Questions from Presentation}

\begin{enumerate}
    \item Why are prices type-specific? 
    \begin{itemize}
        \item Really, this comes from the fact that the final good producer has a downward-sloping demand function for each variety \( \theta \): 
        \[y\left( \theta \right) = \frac{Y}{p\left( \theta \right)^\varepsilon}\]
        At the planner's optimum, \( k(\theta) \) will be strictly increasing in \( \theta \), so \( y(\theta) \) (and thus \( p(\theta) \)) will be unique. We should emphasize this.\footnote{In the competitive equilibrium, prices will be such that \( \theta p\left( \theta \right) = R \) for all \( \theta \), so prices in this case will be unique as well, as we will have \( p\left( \theta \right) = R/\theta \).}
    \end{itemize}
\end{enumerate}

\section{Ariel}
\begin{enumerate}
    \item What facts about heterogeneity in asset returns are we trying to capture?
    \begin{itemize}
        \item Is it heterogeneity in \textit{entrepreneurial outputs}, e.g. Zuckerberg and Gates? 
        \begin{itemize}
            \item I.e. real output
            \item If so, elasticity argument makes sense
            \item As does persistence in infinite horizon case
        \end{itemize} 
        \item Or, is it heterogeneity in \textit{asset returns/financial income}, e.g. David Tepper and other fund managers? 
        \begin{itemize}
            \item In this case, the idea of imperfect substitutability between output is perhaps less believable 
        \end{itemize} 
        \item My thought: the former is more what we have in mind 
        \begin{itemize}
            \item The households who are ``heterogeneous'' are producing capital goods, not generating income based on the outputs of other firms 
            \item It might be helpful to go to something like the SCF, to get some stylized facts on the degree of heterogeneity in asset returns/entrepreneurial income. 
        \end{itemize}
    \end{itemize} 
    \item How do we rationalize our assumption that \( p(\theta) \) is observable to the market, but not the government? 
    \begin{itemize}
        \item This will be important when we consider implementation, as it asks on which observables the government is able to condition capital taxes 
    \end{itemize} 
    \item \label{it:diff} In reality, we have differential taxation. What does our paper have to say about the existing tax code? 
    \begin{itemize}
        \item Is the existing degree of differentiation optimal? 
    \end{itemize} 
    \item Related to (\ref{it:diff}): other papers (mentioned in literature review) have generated differential rates. What do we \textit{add} to this? 
    \begin{itemize}
        \item Similarly, is the main contribution here the theory, or the empirical implications? 
    \end{itemize} 
    \item A subtler discussion of the inefficiencies generated by price-taking is needed. 
    \begin{itemize}
        \item In most macro models, it is the case that individual agents do not internalize the effect of their decisions on prices. This is not inefficient!
        \item To be more clear: when a type \( \theta \) scales up, this affects prices for \textit{everyone} who produces an intermediate good, which creates an externality. 
        \item In Dixit-Stiglitz notation, if I scale up, I change not only \( p_i \), but also the price index \( P \), which affects the demand function for \textit{every} variety. 
    \end{itemize}
\end{enumerate}

\end{document}