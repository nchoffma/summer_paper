\documentclass[11pt]{article}
\usepackage[margin = 1in]{geometry}
\usepackage{amsmath}
\usepackage{amssymb}
\usepackage{amsthm} % for proof environment
\usepackage{enumitem}
\usepackage{graphicx}
\usepackage{indentfirst}
\usepackage{caption}
\usepackage{lscape}
\usepackage{multirow}
\usepackage{array}
\usepackage{setspace}
\setlist{nolistsep}
\usepackage[round]{natbib}
\usepackage{accents}
\usepackage{caption}
\usepackage{subcaption}
\usepackage{xcolor}
\usepackage{setspace}
\onehalfspacing

\title{Literature Review: Dynamic Optimal Taxation}
\author{}
\date{}

\begin{document}
    
\paragraph{Summary:} The goal of this document is to summarize previous findings from the literature on dynamic optimal (Mirrleesian) taxation. Within this, there are two sub-goals:
\begin{enumerate}
    \item Characterize the existing intuition on dynamic capital (income) taxation, to note what changes when we assume heterogeneous \textit{returns}, rather than labor productivities. 
    \item Characterize the existing intuition on dynamic labor income taxes, to highlight the similarities and contrasts between the intuitions behind taxing these sources of income. 
\end{enumerate}

\section{\cite{albanesi2006dynamic}}

\begin{itemize}
    \item The focus here is mainly on the implementation of constrained-efficient allocations in a decentralized economy. 
    \item As in other studies, such as \cite{golosov2003optimal} (section \ref{sec:golosov03}), \cite{albanesi2006dynamic} find the optimality of a positive intertemporal (savings) wedge 
    \begin{itemize}
        \item Intuition: in order to ensure that the proper labor effort is supplied, savings should be taxed in order to prevent agents from self-insuring against future labor income shocks
    \end{itemize}
\end{itemize}


\section{\cite{shourideh2014optimal}}
\begin{itemize}
    \item A proposition analogous to Proposition 2 in \cite{shourideh2014optimal} appears in our current paper. There are three implications of Proposition 2: 
    \begin{enumerate}
        \item The long-run wealth distribution has a Pareto tail
        \item WE*edges independent of history
        \item Consumption of old related to promised utility in a history-independent way
    \end{enumerate}
    Points (2) and (3) do not apply to our model, as we have dynamic complementarities and infinitely-lived agents. Point (1) may apply. 
    \item Optimal savings tax: subsidize savings of old, tax savings of young 
    \begin{itemize}
        \item Savings of young \textit{tightens} the incentive problem, because it offers insurance against negative shock in future 
        \item Savings of old \textit{relaxes} incentive problem, as it confers more consumption to their descendants, relaxing the descendants' incentive constraint.
    \end{itemize} 
    \item Progressive bequest subsidy echoes result in \cite{farhi2010progressive}: should subsidize bequests, which is a distortion, and distort the decisions of more productive agents \textit{less}
    \begin{itemize}
        \item Counter to \cite{golosov2003optimal}: in this model, saving \textit{increases} resources in the future (by relaxing the incentive constraint of the future generation), while in the models of labor income, saving \textit{decreases} future resources (through diminished labor effort)
    \end{itemize}
\end{itemize}


\section{\cite{kocherlakota2005zero}}
\begin{itemize}
    \item Optimal wealth tax is zero in expectation, and regressive: high for surprisingly low-skilled agents, and low for surprisingly high-skilled. 
    \begin{itemize}
        \item Intuition is the same as before: in order to ensure efficient exertion of labor effort, need to deter agents from carrying a high level of wealth into the next period. 
    \end{itemize}
    \item Crucial distinction: because utility is additively separable between consumption and labor, marginal utilities and IMRS are publicly observable, and private information does not affect either. In our paper, this is not the case. 
\end{itemize}

\section{\cite{golosov2003optimal}} \label{sec:golosov03}
\begin{itemize}
    \item Main result is \textit{inverse Euler equation}
    \item Intuition is as usual: savings reduce the dependence of \( t+1 \) consumption on \( t+1 \) skill level, which tightens the incentive problem. Thus, savings should be taxed.
    \item 
\end{itemize}

\section{\cite{farhi2010progressive}}
\begin{itemize}
    \item In dynamic model, \cite{farhi2010progressive} discuss an implementation using linear inheritance taxes (levied on heirs, rather than estates). 
    \item Nests \cite{kocherlakota2005zero} ``zero expected wealth tax'' result as a special case in which the utility of descendants valued \textit{only} through altruism of parents (\( \nu_t = 0 \))
    \begin{itemize}
        \item Otherwise, expected inheritance taxes not zero 
    \end{itemize}
    \item Intuition: as in the static model, the ``progressive subsidies'' on bequests follows from insurance motives 
    \begin{itemize}
        \item If the planner values utility of future generations in above and beyond altruism of ancestors, she wants to insure them against the risk of being born into a poor dynasty 
        \item Consumption across generations is \textit{mean-reverting} (think ``squeezing'' consumption from \( t \) to \( t+1 \))
    \end{itemize}
\end{itemize}

\section{\cite{albanesi2006optimal}}
\begin{itemize}
    \item Individual intertemporal wedge on entrepreneurial capital is given by 
    \begin{align*}
        IW_{K}&=\beta E_{1}u^{\prime}\left(c_{1}^{*}\left(x\right)\right)\left(1+x\right)-u^{\prime}\left(c_{0}^{*}\right)\\&=IW+\beta Cov_{1}\left(u^{\prime}\left(c_{1}^{*}\left(x\right)\right),x\right)
    \end{align*}
    We know that \( Cov_{1}\left(u^{\prime}\left(c_{1}^{*}\left(x\right)\right),x\right)<0 \), so the above shows that \( IW_{K}<IW \), and can take either sign. 
    \item The first term is the marginal benefit of a small increase in entrepreneurial capital, and the second is the marginal cost. 
    \item So, if \( IW_{K}<0 \), then the marginal benefit outweighs the marginal cost, whereas if \( IW_{K}>0 \), the reverse is true. 
    \item  \( IW_{K}>0 \): there is a shadow cost to increasing entrepreneurial capital, as the additional capital tightens the entrepreneurs incentive constraint (adverse effect on incentives)
    \item \( IW_{K}<0 \): there is a shadow benefit to increasing capital, as it relaxes the entrepreneur's incentive constraint. 
    \item The wedge \( IW_K \) can be further decomposed: 
    \begin{equation*}
        IW_{K}=\mu\left(\pi\left(1\right)-\pi\left(0\right)\right)\beta\left\{ \underbrace{\left[u^{\prime}\left(c_{1}^{*}\left(\underline{x}\right)\right)-u^{\prime}\left(c_{1}^{*}\left(\overline{x}\right)\right)\right]\left(1-x\right)}_{\text{Wealth Effect}}-\underbrace{\left(\overline{x}-\underline{x}\right)u^{\prime}\left(c_{1}^{*}\left(\underline{x}\right)\right)}_{\text{Subs. Effect}}\right\} 
    \end{equation*} 
    \item The wealth effect captures the negative effect of capital on incentives: additional wealth increases consumption in the bad state, thereby providing insurance and reducing incentives
    \begin{itemize}
        \item Determined by the difference in marginal utilities across good and bad states
    \end{itemize} 
    \item The substitution effect captures the positive effect of capital on incentives: additional effort increases the likelihood of a good return, which increases effort at higher levels of capital
    \begin{itemize}
        \item Driven by difference in returns \( \left(\overline{x}-\underline{x}\right) \)
    \end{itemize} 
    \item \textbf{Our Model}: we do have wedges \( \tau_k \) that can take either sign!
\end{itemize}
\section{\cite{scheuer2014entrepreneurial}}
Main interest is in \( \S 5 \): Entrepreneurial Taxation with Credit Market Frictions
\begin{itemize}
    \item Pooling equilibrium in the credit market leads to cross-subsidization, which in turn leads to too many low-skilled and too few high-skilled types selecting into entrepreneurship 
    \item Regressive profit tax can correct for this, as it raises the returns to entrepreneurship for higher types. 
    \item \textbf{Our paper}: We don't have the selection effect (all types have \( \theta p(\theta) > R\)), nor do we have these pooling contracts with the possibility of bankruptcy. 
    \begin{itemize}
        \item What we may have, though, is a mix of over- and under-supplying of capital (intensive rather than extensive margin) 
        \item It is possible that in the absence of taxes, high types would over-invest, and low types would under-invest, throwing off the pricing schedule 
        \item It could be that due to complimentarities, the planner wants to ``squeeze'' investment choices across \( \Theta \), as in \cite{farhi2010progressive}
    \end{itemize}
\end{itemize}

\section{Summary}

What lessons can we take from these? 
\begin{itemize}
    \item \cite{farhi2010progressive}: Want to think about taxes as affecting rates of return 
    \begin{itemize}
        \item In the simple two-period model, the taxes ``squeeze'' the rates of return: because subsidies are progressive, the low-type agents enjoy a \textit{higher} post-tax rate of return than do the high-type agents. 
        \item So in this sense, taxes are for redistribution 
        \item Note though, that this only holds if the planner attaches a higher weight to the utility of children than do their parents. 
    \end{itemize} 
    \item Departure of \cite{shourideh2014optimal} from others (\cite{golosov2003optimal}, \cite{albanesi2006dynamic}, etc.)
    \begin{itemize}
        \item The key distinction is whether further investment \textit{relaxes} or \textit{tightens} the incentive constraints 
        \item In most cases with labor income and common (pre-tax) return, further investment tightens the incentive constraint, as the agent has less incentive to work 
        \item In the \cite{shourideh2014optimal} model, however, it can operate in both directions 
        \item Savings of old subsidized (progressively), for Farhi-Werning reasons
        \item \textbf{Our model}: I think we have a pull in both directions. 
    \end{itemize}
\end{itemize}

\bibliographystyle{named}
\bibliography{dynamic_lit_review}

\end{document}