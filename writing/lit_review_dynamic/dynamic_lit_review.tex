\documentclass[11pt]{article}
\usepackage[margin = 1in]{geometry}
\usepackage{amsmath}
\usepackage{amssymb}
\usepackage{amsthm} % for proof environment
\usepackage{enumitem}
\usepackage{graphicx}
\usepackage{indentfirst}
\usepackage{caption}
\usepackage{lscape}
\usepackage{multirow}
\usepackage{array}
\usepackage{setspace}
\setlist{nolistsep}
\usepackage[round]{natbib}
\usepackage{accents}
\usepackage{caption}
\usepackage{subcaption}
\usepackage{xcolor}
\usepackage{setspace}
\onehalfspacing

\title{Literature Review: Dynamic Optimal Taxation}
\author{}
\date{}

\begin{document}
    
\paragraph{Summary:} The goal of this document is to summarize previous findings from the literature on dynamic optimal (Mirrleesian) taxation. Within this, there are two sub-goals:
\begin{enumerate}
    \item Characterize the existing intuition on dynamic capital (income) taxation, to note what changes when we assume heterogeneous \textit{returns}, rather than labor productivities. 
    \item Characterize the existing intuition on dynamic labor income taxes, to highlight the similarities and contrasts between the intuitions behind taxing these sources of income. 
\end{enumerate}

\section{\cite{albanesi2006dynamic}}

\begin{itemize}
    \item The focus here is mainly on the implementation of constrained-efficient allocations in a decentralized economy. 
    \item As in other studies, such as \cite{golosov2003optimal} (section \ref{sec:golosov03}), \cite{albanesi2006dynamic} find the optimality of a positive intertemporal (savings) wedge 
    \begin{itemize}
        \item Intuition: in order to ensure that the proper labor effort is supplied, savings should be taxed in order to prevent agents from self-insuring against future labor income shocks
    \end{itemize}
\end{itemize}


\section{\cite{shourideh2014optimal}}
\begin{itemize}
    \item A proposition analogous to Proposition 2 in \cite{shourideh2014optimal} appears in our current paper. There are three implications of Proposition 2: 
    \begin{enumerate}
        \item The long-run wealth distribution has a Pareto tail
        \item Wedges independent of history
        \item Consumption of old related to promised utility in a history-independent way
    \end{enumerate}
    Points (2) and (3) do not apply to our model, as we have dynamic complementarities and infinitely-lived agents. Point (1) may apply. 
    \item Optimal savings tax: subsidize savings of old, tax savings of young 
    \begin{itemize}
        \item Savings of young \textit{tightens} the incentive problem, because it offers insurance against negative shock in future 
        \item Savings of old \textit{relaxes} incentive problem, as it confers more consumption to their descendants, relaxing the descendants' incentive constraint.
    \end{itemize} 
    \item Progressive bequest subsidy echoes result in \cite{farhi2010progressive}: should subsidize bequests, which is a distortion, and distort the decisions of more productive agents \textit{less}
    \begin{itemize}
        \item Counter to \cite{golosov2003optimal}: in this model, saving \textit{increases} resources in the future (by relaxing the incentive constraint of the future generation), while in the models of labor income, saving \textit{decreases} future resources (through diminished labor effort)
    \end{itemize}
\end{itemize}


\section{\cite{kocherlakota2005zero}}
\begin{itemize}
    \item Optimal wealth tax is zero in expectation, and regressive: high for surprisingly low-skilled agents, and low for surprisingly high-skilled. 
    \begin{itemize}
        \item Intuition is the same as before: in order to ensure efficient exertion of labor effort, need to deter agents from carrying a high level of wealth into the next period. 
    \end{itemize}
    \item Crucial distinction: because utility is additively separable between consumption and labor, marginal utilities and IMRS are publicly observable, and private information does not affect either. In our paper, this is not the case. 
\end{itemize}

\section{\cite{golosov2003optimal}} \label{sec:golosov03}
\begin{itemize}
    \item Main result is \textit{inverse Euler equation}
    \item Intuition is as usual: savings reduce the dependence of \( t+1 \) consumption on \( t+1 \) skill level, which tightens the incentive problem. Thus, savings should be taxed.
\end{itemize}

\section{\cite{farhi2010progressive}}
\begin{itemize}
    \item In dynamic model, \cite{farhi2010progressive} discuss an implementation using linear inheritance taxes (levied on heirs, rather than estates). 
    \item Nests \cite{kocherlakota2005zero} ``zero expected wealth tax'' result as a special case in which the utility of descendants valued \textit{only} through altruism of parents (\( \nu_t = 0 \))
    \begin{itemize}
        \item Otherwise, expected inheritance taxes not zero 
    \end{itemize}
    \item Intuition: as in the static model, the ``progressive subsidies'' on bequests follows from insurance motives 
    \begin{itemize}
        \item If the planner values utility of future generations in above and beyond altruism of ancestors, she wants to insure them against the risk of being born into a poor dynasty 
        \item Consumption across generations is \textit{mean-reverting} (think ``squeezing'' consumption from \( t \) to \( t+1 \))
    \end{itemize}
\end{itemize}

\section{Summary}

What lessons can we take from these? 
\begin{itemize}
    \item \cite{farhi2010progressive}: Want to think about taxes as affecting rates of return 
    \begin{itemize}
        \item In the simple two-period model, the taxes ``squeeze'' the rates of return: because subsidies are progressive, the low-type agents enjoy a \textit{higher} post-tax rate of return than do the high-type agents. 
        \item So in this sense, taxes are for redistribution 
        \item Note though, that this only holds if the planner attaches a higher weight to the utility of children than do their parents. 
    \end{itemize} 
    \item Departure of \cite{shourideh2014optimal} from others (\cite{golosov2003optimal}, \cite{albanesi2006dynamic}, etc.)
    \begin{itemize}
        \item The key distinction is whether further investment \textit{relaxes} or \textit{tightens} the incentive constraints 
        \item In most cases with labor income and common (pre-tax) return, further investment tightens the incentive constraint, as the agent has less incentive to work 
        \item In the \cite{shourideh2014optimal} model, however, it can operate in both directions 
        \item Savings of old subsidized (progressively), for Farhi-Werning reasons
        \item \textbf{Our model}: I think we have a pull in both directions. 
    \end{itemize}
\end{itemize}

\bibliographystyle{named}
\bibliography{dynamic_lit_review}

\end{document}