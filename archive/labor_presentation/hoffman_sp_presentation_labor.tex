\documentclass{beamer}
\usepackage[utf8]{inputenc}
\usepackage{hyperref} 
\usepackage{pgfpages}
\usepackage{amsmath,amssymb,mathrsfs}
\setbeamertemplate{navigation symbols}{} % turns off navigation symbols at the bottom of slides
\usepackage{graphicx}

\usepackage{caption}
\usepackage{subcaption}
\usepackage[round]{natbib}

\title{Summer Paper: Optimal Taxation with Idiosyncratic Return Shocks}
\author[Hoffman, Shourideh]
{Nick Hoffman\inst{1} \and Ali Shourideh \inst{1}}
\institute[ ] 
{\inst{1} Carnegie Mellon University}

\begin{document}
    \begin{frame}
        \titlepage 
    \end{frame}

\begin{frame}
    \frametitle{}
    \tableofcontents
\end{frame}

\section{Main Research Questions}
\begin{frame}
    \frametitle{Main Question(s)}

    \begin{enumerate}
        \item What is the optimal tax schedule in an economy in which agents face idiosyncratic income and return shocks?
        \begin{enumerate}
            \item Is there a role for a tax on wealth?
        \end{enumerate}
        \item Does this schedule change with the introduction of non-pecuniary motives for wealth accumulation? (time permitting)
    \end{enumerate}

\end{frame}
\section{Literature Review}

\begin{frame}
    \frametitle{Return Shocks}

    \begin{itemize}
        \item Heterogeneous-agent models in the vein of \cite{aiyagari1994uninsured}, \cite{huggett1996wealth} are unable to capture empirical distribution of wealth 
        \item One augmentation: idiosyncratic shocks to rates of return 
        \begin{itemize}
            \item \cite{benhabib2011distribution}, \cite{benhabib2015wealth}: Stationary distribution of wealth in these models has a Pareto tail, as in the data 
            \item \cite{benhabib2019wealth}: evidence that these shocks can help match social mobility, another appealing feature 
        \end{itemize}
    \end{itemize}

\end{frame}

\begin{frame}
    \frametitle{Optimal Income Taxation}
    
    \cite{mirrlees1971exploration}
    \begin{itemize}
        \item Formal characterization of the tradeoff between efficiency and redistributive motives
        \item Problem: government sets optimal tax schedule to maximize social welfare function
        \item Solution must satisfy budget and individual rationality constraints
        \item Government can only observe income, not type, and thus this becomes a signalling problem
        \item Incentivize agents to reveal their types 
        \item \cite{diamond1998optimal}, \cite{saez2001using}: shape of optimal taxes depends on functional assumptions; case for progressive taxes exists
    \end{itemize}

\end{frame}


\begin{frame}
    \frametitle{Capital Taxation}

    \begin{itemize}
        \item Great deal of attention paid to labor income taxation, and the progressivity thereof
        \item Classic result: optimal tax on capital is 0
        \item \cite{saez2019triumph}: cannot achieve desired progressivity with labor income tax alone
        \begin{itemize}
            \item Trivial example: think of famous CEOs % need to make sure labor income & cap gains tax are in concert
        \end{itemize}
        \item Motivation for considering capital income and wealth tax: how to tax wealthy individuals?
        \item Mirrleesian model gives us a framework to consider the tradeoffs. 
        \begin{itemize}
            \item Some examples: \cite{golosov2003optimal}, \cite{albanesi2006dynamic}, \cite{golosov2006new}
        \end{itemize}
    \end{itemize}

\end{frame}

\section{Research Plan}

\begin{frame}
    \frametitle{Capital Taxation} % how to apply Mirrlees to wealth 

    \begin{itemize}
        \item In the classical \cite{mirrlees1971exploration} model, individuals choose their effort level, given the tax schedule of the government
        \item Here, individuals will choose their savings rates and risk-taking behavior.
        \item The government's problem: would like to reward risk-taking and discipline, but cannot observe history of earnings shocks   
        \begin{itemize}
            \item Constraint is now to ensure that individuals are incentivized to put wealth towards productive purposes % opportunity for non-pecuniary motives?
        \end{itemize}
        % Good place to put some citations from the user's guide
    \end{itemize}

\end{frame}

\begin{frame}
    \frametitle{Model Features}

    \begin{itemize}
        \item Agents with preferences over consumption and liesure
        \begin{itemize}
            \item Allocate wealth and income between consumption and savings 
        \end{itemize}
        \item Idiosyncratic, persistent schocks to income and rates of return, calibrated to data 
        \begin{itemize}
            \item  \cite{floden2001idiosyncratic} use PSID for income process
            \item PSID wealth supplements can be used to study rates of return on various components of household wealth (albeit with limited scope)
            \item \cite{benhabib2011distribution} demonstrates that there must be persistence to wealth accumulation process to get stationary distribution
        \end{itemize}
    \end{itemize}

\end{frame}

\begin{frame}
    \frametitle{Model Features}

    \begin{itemize}
        \item Distribution of ability, unobserved by the policymaker
        \begin{itemize}
            \item \cite{diamond1998optimal} and \cite{saez2001using} suggest that this distribution should be Pareto 
        \end{itemize} 
        \item Government
        \begin{itemize}
            \item Levies taxes based on observable characteristics (income, wealth)
            \item Maximizes a social welfare function 
            \item Coordinates the tax schedule to satisfy individual rationality constraints and meet its budgetary requirements
        \end{itemize}
    \end{itemize}

\end{frame}

\begin{frame}
    \frametitle{Potential Extension}

    Non-pecuniary Motives
    \begin{itemize}
        \item Key question: why are savings rates high among wealthy individuals?
        \item Various studies have suggested motives for saving outside of precautionary motive
        \begin{itemize}
            \item Bequest motive 
            \item Conspicuous consumption
            \item \cite{genicot2017aspirations}: individuals form \textit{aspirations}, threshold values of wealth. Crossing these gives additional utility.
            \item These motives are relevant for optimal taxation insofar as they affect elasiticites
        \end{itemize}
    \end{itemize}

\end{frame}

\begin{frame}[allowframebreaks]
    \frametitle{References}

    \bibliographystyle{named}
    \bibliography{summer_paper}

\end{frame}
\end{document}