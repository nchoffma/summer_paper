\documentclass[11pt]{article}
\usepackage[margin = 1in]{geometry}
\usepackage{amsmath}
\usepackage{amssymb}
\usepackage{amsthm} % for proof environment
\usepackage{enumitem}
\usepackage{graphicx}
\usepackage{indentfirst}
\usepackage{caption}
\usepackage{lscape}
\usepackage{multirow}
\usepackage{array}
\usepackage{subcaption} % caption the subfigures
\usepackage{setspace}
\setlist{nolistsep}
\usepackage[round]{natbib}

\begin{document}
\begin{flushleft}
    Presentation Outline \\
    \today
\end{flushleft}

    \begin{itemize}
        \item Main Research Question(s)
        \begin{enumerate}
            \item What is the optimal tax schedule in an economy where individuals face idiosyncratic shocks to both income, and rates of return on wealth?
            \item Does the optimal tax schedule change when non-pecuniary motives for wealth accumulation are added?
        \end{enumerate}

        \item Summary of Related Literature
        \begin{itemize}
            \item Optimal Taxation
            \begin{itemize}
                \item \cite{mirrlees1971exploration}
                \begin{itemize}
                    \item Seminal paper in optimal taxation
                    \item Formal characterization of the tradeoff between efficiency and redistributive motives
                    \item Problem: government sets optimal tax schedule to maximize social welfare function
                    \item Must satisfy its budget constraint, along with individual rationality constraint
                    \item Government can only observe income, not type
                    \item One finding: optimal top marginal rate is zero
                \end{itemize}
                \item \cite{diamond1998optimal}
                \begin{itemize}
                    \item Alternative characterization of first-order conditions of the \cite{mirrlees1971exploration}
                    \item Stronger case for very progressive taxation 
                    \item Top rate of zero: minimal bearing on policy; rate need not slowly decline to zero 
                    \item Generally: optimal marginal rate at a given income level depends on the elasticity of labor supply around this level (did Mirrlees show this?)
                \end{itemize}
                \item \cite{saez2001using}
                \begin{itemize}
                    \item Building on \cite{diamond1998optimal}, uses elaticities observed in data to derive optimal marginal rate
                \end{itemize}
            \end{itemize}
            \item Taxation of capital and capital income 
            \begin{itemize}
                \item \cite{feldstein1978welfare}
                \item \cite{saez2019triumph}: meeting optimal progressivity requires taxes other than on income 
                \item Cite some Saez/Zucman/Piketty things
            \end{itemize}
            
            \item Idiosyncratic RoR shocks
            \begin{itemize}
                \item \cite{benhabib2011distribution}, \cite{benhabib2015wealth}, \cite{benhabib2019wealth}, \cite{gabaix2016dynamics}
                \item Pareto tail: need persistence to (a) have stationary dist and (b) match data
            \end{itemize}
        \end{itemize}

        \item Early Stages of Model
        \item 
    \end{itemize}

\bibliographystyle{named}
\bibliography{summer_paper}
\end{document}
